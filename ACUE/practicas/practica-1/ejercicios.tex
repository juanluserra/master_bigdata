% Clast de documento
\documentclass[12pt, letterpaper]{article}

% Paquetes
\usepackage[utf8]{inputenc}
\usepackage[spanish]{babel}
\usepackage{biblatex}
\usepackage{csquotes}
\usepackage{datetime}
\usepackage{lipsum}
\usepackage{hyperref}
\usepackage{fancyhdr}
\usepackage{parskip}
\usepackage{amsmath}
\usepackage{listings}
\usepackage{xcolor} 
\usepackage{booktabs}

%------------ 
% Decoración
%------------

% Configuración de colores de python
\lstset{
    language=Python,
    basicstyle=\ttfamily\small,
    keywordstyle=\color{blue},
    commentstyle=\color{green},
    stringstyle=\color{red},
    showstringspaces=false,
    numbers=left,
    numberstyle=\tiny\color{gray},
    frame=single,
    breaklines=true
}

\lstset{
    literate={á}{{\'a}}1 {é}{{\'e}}1 {í}{{\'\i}}1 {ó}{{\'o}}1 {ú}{{\'u}}1
             {Á}{{\'A}}1 {É}{{\'E}}1 {Í}{{\'I}}1 {Ó}{{\'O}}1 {Ú}{{\'U}}1
             {ñ}{{\~n}}1 {Ñ}{{\~N}}1
             {¡}{{!`}}1 {¿}{{?`}}1 % Añade caracteres especiales en español
}

% Configuración de header y footer
\fancyhf{}
\setlength{\headheight}{15.71667pt}
\addtolength{\topmargin}{-3.71667pt}
\fancyhf{}

% Header
\fancyhead[L]{\textsc{\doctitle}}
\renewcommand{\sectionmark}[1]{\markright{#1}}
\fancyhead[R]{\textit{\nouppercase{\rightmark}}}

% Footer
\renewcommand{\footrulewidth}{0.4pt}
\fancyfoot[C]{Página \thepage}

% Título
\newcommand{\doctitle}{Ejercicios práctica 1}
\title{\doctitle}
\author{Juan Luis Serradilla Tormos}
\date{\monthname[\month] de \the\year}

% Bibliografía
\addbibresource{test.bib}

% Eliminar sangría
\setlength{\parindent}{0pt}

% Aumentar la separación entre párrafos
\setlength{\parskip}{1em plus 0.5em minus 0.2em}


%-----------
% Documento
%-----------
\begin{document}

% Mostrar header y footer
\pagestyle{fancy}

% Mostrar el título
\maketitle

% Índice
\newpage
\tableofcontents

% Contenido
\newpage
\section{Ejercicio 1}
\textbf{Una tienda de abarrotes recibe su suministro semanal de
huevos cada jueves por la mañana. Este envío debe durar hasta el jueves
siguiente, cuando llega un nuevo pedido. Cualquier huevo no vendido antes
del jueves se descarta. Los huevos se venden a \$10 por cada cien y cuestan
\$8 por cada cien. La demanda semanal de huevos en esta tienda variaba de una
semana a otra. Basándose en la experiencia pasada, se asigna la siguiente
distribución de probabilidad a la demanda semanal:}
\begin{table}[h]
    \centering
    \begin{tabular}{cccccc}
        \toprule
        \textbf{Demanda (x100 huevos)} & 10 & 11 & 12 & 13 & 14 \\
        \midrule
        \textbf{Probabilidad de demanda} & 0.1 & 0.2 & 0.4 & 0.2 & 0.1 \\
        \bottomrule
    \end{tabular}
\end{table}
\textbf{Este patrón de demanda se mantiene estable durante todo el año; la
demanda de huevos no es estacional y la tendencia es plana. El problema
es: ¿Cuántos huevos deben ordenarse para la entrega cada jueves?}

Para poder ver cuántos huevos se deben ordenar lo primero que hay que hacer es determinar una función beneficio.
\begin{align*}
    & n \equiv \text{Número de huevos ordenados} \\ 
    & d \equiv \text{Demanda semanal} \\
    & p \equiv \text{probabilidad de demanda} \\
    & c \equiv \text{Costo de los huevos} \\
    & v \equiv \text{Valor de los huevos} 
\end{align*}

La variable $d_i$ representa la demanda semanal de la semana $i$, $p_i$ la probabilidad de que ocurra dicha demanda y $n_j$  la cantidad de huevos que se encargan, tal que $i,j \in \{1,2,3,4,5\}$. La cantidad de huevos que se encargan oscilan entre los valores de $d$ ya que si se piden menos del $\min{d}$ se perderán ventas y si se piden más del $\max{d}$ se dejarán huevos sin vender. Por tanto, la función beneficio se puede expresar como:
\[
    B(n_j,d_i) = \begin{cases}
        n_j \cdot (v - c) & \text{si } j \leq i \\
        d_i \cdot v - n_j \cdot c & \text{si } j > i
    \end{cases}
\]

\newpage
Ahora que tenemos la función beneficio, tenemos que calcular el beneficio medio para una cantidad $n_j$. Para ello, se calcula el beneficio medio ponderado por la probabilidad de que ocurra una demanda $d_i$:
\[
    \bar{B}(n_j) = \sum_{i=1}^{5} p_i \cdot B(n_j,d_i) = 
    \sum_{i<j}p_i(d_i \cdot v - n_j \cdot c) + \sum_{i \geq j}p_i(n_j \cdot (v - c))
\]

Reordenando y juntando términos, se obtiene:
\[
    \bar{B}(n_j) = 
        v\sum_{i<j}p_i d_i + 
        v n_j \sum_{i \geq j} p_i -
        c n_j
\]

Ahora, vamos a aplicar la función beneficio a cada posible valor de $j$.
\begin{align*}
    & j = 1 \Rightarrow B(10) = 20 \\
    & j = 2 \Rightarrow B(11) = 21 \\
    & j = 3 \Rightarrow B(12) = 20 \\
    & j = 4 \Rightarrow B(13) = 15 \\
    & j = 5 \Rightarrow B(14) = 8
\end{align*}

Finalmente, vemos que el valor óptimo de $n$ es $n = 11$, que da un beneficio medio de 21\$.

\newpage
\section{Ejercicio 2}
\textbf{La función de utilidad de Ana es $U = \sqrt{w}$, donde $w$ es el
dinero. Ella es dueña de una panadería, la cual el año que viene puede valer
o bien 0 o bien 100 euros con las mismas posibilidades.}
\begin{enumerate}
    \item \textbf{Supongamos que su empresa es el único activo que posee. ¿Cuál es el precio P más bajo al que aceptaría vender su panadería?}

    La función de utilidad es $U = \sqrt{w}$, por lo que la utilidad media es $\bar{U} = \sum_i p_i\sqrt{w_i}$. Sustituyendo tenemos $\bar{U} = \frac{1}{2}(\sqrt{0} + \sqrt{100}) = \frac{1}{2}(0 + 10) = 5$. 

    Ahora que tenemos la utilidad media, si despejamos el precio medio como $\bar{w} = \bar{U}^2$ obtenemos que $\bar{w} = 5^2 = 25$. Por tanto, el precio más bajo al que aceptaría vender su panadería es 25 euros.

    \item \textbf{Supongamos que tiene tiene 100 euros guardados ¿Cómo afectaría en este caso al caso anterior?}
    
    En este caso tenemos que $\bar{U} = \sum_i p_i\sqrt{100 + w_i}$, por lo que sustotuyendo tenemos $\bar{U} = \frac{1}{2}(\sqrt{100 + 0} + \sqrt{100 + 100}) = \frac{1}{2}(10 + 14.14) = 12.07$. 

    Si despejamos la utilidad media como $\bar{w} = \bar{U}^2 - 100$ obtenemos que $\bar{w} = 12.07^2 - 100 = 45.7$. Por tanto, el precio más bajo al que aceptaría vender su panadería es 45.7 euros.

    \item\textbf{Compara tus resultados de las partes anteriores. ¿Cuál es la relación entre los ingresos de Ana y su aversión al riesgo?}
    
    La aversión al riesgo se calcula como $A(w) = -\frac{d^2U}{dw^2} {\big(\frac{dU}{dw}\big)}^{-1}$. Calculando las derivadas tenemos:
    \begin{align*}
        &   \frac{dU}{dw} = \frac{1}{2\sqrt{w}},\
        \frac{d^2U}{dw^2} = -\frac{1}{4w\sqrt{w}} \Rightarrow
        A(w) = \frac{1}{2\sqrt{w}}
    \end{align*}

    Podemos ver que la aversión al riesgo disminuye cuando aumenta el dinero.

\end{enumerate}

\end{document}
