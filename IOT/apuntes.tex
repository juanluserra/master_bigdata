
% Clase de documento
\documentclass[12pt, letterpaper]{article}

% Paquetes
\usepackage[utf8]{inputenc}
\usepackage[spanish]{babel}
\usepackage{biblatex}
\usepackage{csquotes}
\usepackage{datetime}
\usepackage{lipsum}
\usepackage{hyperref}
\usepackage{parskip}
\usepackage{fancyhdr}

%------------ 
% Decoración
%------------
\setlength{\headheight}{15.71667pt}
\addtolength{\topmargin}{-3.71667pt}
\fancyhf{}

% Header
\fancyhead[L]{\textsc{\doctitle}}
\renewcommand{\sectionmark}[1]{\markright{#1}}
\fancyhead[R]{\textit{\nouppercase{\rightmark}}}

% Footer
\renewcommand{\footrulewidth}{0.4pt}
\fancyfoot[C]{Página \thepage}

% Título
\newcommand{\doctitle}{Apuntes de IoT}
\title{\doctitle}
\author{Juan Luis Serradilla Tormos}
\date{\monthname[\month] de \the\year}

% Bibliografía
\addbibresource{test.bib}

% Eliminar sangría
\setlength{\parindent}{0pt}

% Aumentar la separación entre párrafos
\setlength{\parskip}{1em plus 0.5em minus 0.2em}

%-----------
% Documento
%-----------
\begin{document}

% Activamos el estilo de página
\pagestyle{fancy}

% Mostrar el título
\maketitle

% Índice
\newpage
\tableofcontents

% Contenido
\newpage

\section{Tema 2}

\subsection{Arquitectura funcional de un nodo sensor/actuador}

\begin{itemize}
    \item \textbf{Bloque sensor/actuador}
    \item \textbf{Bloque de procesado}
    \item \textbf{Bloque de comuniación}
\end{itemize}

\subsection{M2M}
El término M2M se refiere a ``Machine to Machine''. La diferencia entre IoT y M2M es que M2M es solo de equipo a equipo, mientras que en IoT hay una red común de información entre dispositivos.

\subsection{Gateways en una arquitectura IoT}
Un \textit{gateway} en una arquitectura IoT actúa como un puente entre los dispositivos IoT y la red. Su función principal es traducir los diferentes protocolos de comunicación utilizados por los dispositivos IoT a un protocolo común que pueda ser entendido por la red. Además, los \textit{gateways} pueden realizar tareas de procesamiento de datos, filtrado y seguridad antes de enviar la información a la nube o a otros sistemas. Esto permite una comunicación eficiente y segura entre los dispositivos IoT y las aplicaciones que los gestionan.




\end{document}

