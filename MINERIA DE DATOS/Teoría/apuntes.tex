% Clase de documento
\documentclass[12pt, letterpaper]{article}

%----------
% Paquetes
%----------

% Paquetes de codificación y lenguaje
\usepackage[utf8]{inputenc}
\usepackage[spanish]{babel}

% Paquetes de bibliografía
\usepackage{biblatex}

% Paquetes de fecha y hora
\usepackage{datetime}

% Paquetes de contenido
\usepackage{lipsum}

% Paquetes de enlaces
\usepackage{hyperref}

% Paquetes de formato y estilo
\usepackage{fancyhdr}
\usepackage{csquotes}



%------------ 
% Decoración
%------------
\pagestyle{fancy}
\fancyhf{}

% Header
\setlength{\headheight}{15.71667pt}
\fancyhead[L]{\textsc{\doctitle}}
\renewcommand{\sectionmark}[1]{\markright{#1}}
\fancyhead[R]{\textit{\nouppercase{\rightmark}}}

% Footer
\renewcommand{\footrulewidth}{0.4pt}
\fancyfoot[C]{Página \thepage}

% Título
\newcommand{\doctitle}{Apuntes de Minería de Datos}
\title{\doctitle}
\author{Juan Luis Serradilla Tormos}
\date{\monthname[\month] de \the\year}

% Bibliografía
\addbibresource{test.bib}

% Eliminar sangría
\setlength{\parindent}{0pt}


%-----------
% Documento
%-----------
\begin{document}

% Mostrar el título
\maketitle

% Índice
\newpage
\tableofcontents

% Contenido
\newpage
\section{Preprocesamiento de datos}

\subsection{Introducción}
\begin{itemize}
    \item El resultado de la minería de datos depende en gran medida de la calidad de los datos.
    \item El conjunto de datos estará formado por objetos
    \item Los objetos se describen por medio de atributos
    \item Un atributo tiene asociado un tipo que define de los valores que puede tomar
\end{itemize}

\subsection{Limpieza de datos}
Los errores en los datos pueden deberse a diferentes causas:
\begin{itemize}
    \item \textbf{Datos incompletos}: Pueden faltar atributos de interés, valores de los propios\ldots
    \item \textbf{Datos ruidosos}: Datos con ruido o errores, valores duplicados\ldots
    \item \textbf{Datos inconsistentes}: Datos que discrepan en códigos y nombres, en valores duplicados, etc. Por ejemplo.
    \begin{itemize}
        \item Edad = ''42'', Fecha de nacimiento = ''12/07/2015''
        \item Objetos con escala ``1,2,3'' y otros con escala ``A,B,C''
    \end{itemize}
    \item \textbf{Errores intencionados:} Datos que se introducen para encubrir falta de datos. Por ejemplo, encontrar la misma fecha de nacimiento para un gran grupo de personas ya que no tenían fecha 
\end{itemize}

\subsubsection{Datos ausentes}
Los datos ausentes pueden producir varios errores:
\begin{itemize}
    \item \textbf{Pérdida de eficacia:} Se extraen menos patronas y las conclusiones son menos concluyentes.
    \item \textbf{Complicaciones al analizar:} Surgen complicaciones debido a que hay técnicas que no están preparadas para gestionarlos.
    \item \textbf{Sesgo:} Puede haber sesgo en los datos que se extraen.
\end{itemize}

Por todo esto, al limpiar un dataset es necesario tener en cuenta los datos ausentes. Hay varias formas de detectarlos:
\begin{itemize}
    \item Generalmente se representan como valores nulos.
    \item Puede haber nulos camuflados, es decir, valores que no son nulos pero los representan. Esto se debe a que la integridad del sistema no permite introducir nulos en campos con ciertos formatos (direcciones, teléfonos, códigos postales\ldots)
\end{itemize}

\vspace{1em}
\underline{\textbf{Soluciones}}
\vspace{1em}

Vamos a ver soluciones al tratamiento de los datos ausentes:
\begin{itemize}
    \item \textbf{No hacer nada:} Hay métodos (como los árboles de decisión) que son robustos a los datos ausentes.
    \item \textbf{Eliminar los atributos:} Esta es una solución extrema que es necesaria en el alto porcentaje de nulos. En otros casos podemos encontrar un atributo dependiente de mayor calidad.
    \item \textbf{Eliminar el objeto:} Suele hacerse cuando en un problema de clasificación al clase está ausente, pero no es efectivo si el porcentaje de ausentes varía mucho entre atributos. Además, puede introducir sesgo.
    \item \textbf{Reemplazar:} Se puede reemplazar el hueco por un valor. Hay varias formas de hacerlo:
        \begin{itemize}
            \item Manualmente si no hay muchos valores ausentes.
            \item Por un valor que preserve la media o la varianza en datos numéricos, o la moda en datos nominales.
            \item Imputación: consiste en reemplazar estos valores faltantes con estimaciones o valores plausibles, permitiendo que el dataset esté completo y sea útil para el análisis. Hay varias formas de usarla:
                \begin{itemize}
                    \item Usar el valor medio (de todos los valores de los atributos o de solo los que pertenecen a la misma clase).
                    \item Usar el valor más probable.
                    \item Predecir el valor mediante alguna técnica (regresión, KNN, etc).
                \end{itemize}
            \item Mediante técnicas específicas. Por ejemplo, la detección del sexo a través del nombre.
        \end{itemize}
\end{itemize}

A pesar de que soluciones como la imputación sean técnicas muy útiles y frecuentes, hay que tener en cuenta que se sigue perdiendo información, e incluso que el dato que introducimos sea erróneo.

\subsubsection{Datos ruidosos}
Entendemos el ruido como un error o varianza aleatoria en una medición de una variable.

\end{document}
