% Clase de documento
\documentclass[12pt, letterpaper]{article}

%----------
% Paquetes
%----------

% Paquetes de codificación y lenguaje
\usepackage[utf8]{inputenc}
\usepackage[spanish]{babel}

% Paquetes de bibliografía
\usepackage{biblatex}

% Paquetes de fecha y hora
\usepackage{datetime}

% Paquetes de contenido
\usepackage{lipsum}

% Paquetes de enlaces
\usepackage{hyperref}

% Paquetes de formato y estilo
\usepackage{fancyhdr}
\usepackage{csquotes}



%------------ 
% Decoración
%------------
\pagestyle{fancy}
\fancyhf{}

% Header
\setlength{\headheight}{15.71667pt}
\fancyhead[L]{\textsc{\doctitle}}
\renewcommand{\sectionmark}[1]{\markright{#1}}
\fancyhead[R]{\textit{\nouppercase{\rightmark}}}

% Footer
\renewcommand{\footrulewidth}{0.4pt}
\fancyfoot[C]{Página \thepage}

% Título
\newcommand{\doctitle}{Apuntes de Minería de Datos}
\title{\doctitle}
\author{Juan Luis Serradilla Tormos}
\date{\monthname[\month] de \the\year}

% Bibliografía
\addbibresource{test.bib}


%-----------
% Documento
%-----------
\begin{document}

% Mostrar el título
\maketitle

% Índice
\newpage
\tableofcontents

% Contenido
\newpage
\section{Preprocesamiento de datos}

\subsection{Introducción}
\begin{itemize}
    \item El resultado de la minería de datos depende en gran medida de la calidad de los datos.
    \item El conjunto de datos estará formado por objetos
    \item Los objetos se describen por medio de atributos
    \item Un atributo tiene asociado un tipo que define de los valores que puede tomar
\end{itemize}

\subsection{Limpieza de datos}
Los errores en los datos pueden deberse a diferentes causas:
\begin{itemize}
    \item \textbf{Datos incompletos}: Pueden faltar atributos de interés, valores de los propios\ldots
    \item \textbf{Datos ruidosos}: Datos con ruido o errores, valores duplicados\ldots
    \item \textbf{Datos inconsistentes}: Datos que discrepan en códigos y nombres, en valores duplicados, etc. Por ejemplo.
    \begin{itemize}
        \item Edad = ''42'', Fecha de nacimiento = ''12/07/2015''
        \item Objetos con escala ``1,2,3'' y otros con escala ``A,B,C''
    \end{itemize}
    \item \textbf{Errores intencionados:} Datos que se introducen para encubrir falta de datos. Por ejemplo, encontrar la misma fecha de nacimiento para un gran grupo de personas ya que no tenían fecha 
\end{itemize}

\end{document}
